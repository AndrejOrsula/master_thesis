\section{Feature Extraction from RGB and RGB-D Observations}\label{app:feature_extraction_from_rgb_and_rgbd_observations}

This is a network architecture for the feature extractor that was used for RGB and RGB-D observations for \hyperref[subsec:comparison_of_2d_2_5d_3d_observations]{experiment~\ref*{subsec:comparison_of_2d_2_5d_3d_observations}}. It is analogous to octree-based feature extractor from \autoref{subsec:feature_extraction}. For RGB-D observations, the input image contains an additional channel for depth information. All input channels are normalised to range~\(0, 1\), where maximum depth of~2~m is used.

\capstartfalse%
\begin{figure}[ht]
    \centering
    % \includegraphics[width=1.0\textwidth]{}
    TODO: Figure
\end{figure}
\capstarttrue%
