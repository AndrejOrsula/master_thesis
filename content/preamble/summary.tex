\chapter*{Summary}
\addcontentsline{toc}{chapter}{Summary}

In this work, deep reinforcement learning is applied for the task of vision-based robotic grasping with focus on generalisation to diverse objects in varying scenes. Model-free reinforcement learning is employed to learn an end-to-end policy that directly maps visual observations to continuous actions in Cartesian space. For observations, octrees are utilised in a novel approach to provide an efficient representation of the 3D scene. In order to allow agent to generalise over spatial positions and orientations, octree-based 3D convolutional neural network is designed to extract abstract visual features. Agent is then trained by combining such feature extractor with off-policy actor-critic reinforcement learning algorithms.

As training of robotics agents in real world is expensive and potentially unsafe, a new simulation environment for robotic grasping is created. This environment is developed on top of open-source Ignition Gazebo robotics simulator in order to provide high-fidelity physics and photorealistic rendering. Sim-to-real transfer of a learned policy is made possible by combining a dataset of realistic 3D scanned objects and textures with domain randomisation. Among others, this includes randomising the pose of a virtual RGB-D camera with aim to simplify the transfer of a simulated setup to real-world domain.

Results of experimental evaluation indicate that deep reinforcement learning can be applied to learn an end-to-end policy with octree-based observations, while providing noteworthy advantages over traditionally used RGB and RGB-D images. On novel scenes with static camera pose, agent with octree observations is able to reach a success rate of~81.5\%, whereas agent with RGB-D observations and analogous feature extractor achieves~59\%. However, the advantage of 3D observations emerges with invariance to camera pose, where both RGB and RGB-D observations struggle to learn a policy while octrees still retain a success rate of~77\%.

The same policy can be successfully transferred to a real robot without any need for retraining. On scenes with previously unseen real-world everyday objects, a policy trained solely inside simulation can achieve success rate of~68.3\%. The invariance to camera pose enables a simple transfer without requiring the real-world setup to match its digital counterpart. Furthermore, approach from this work allows in some cases to transfer a policy trained on one robot to a different one, while achieving almost identical performance to a policy that was trained on the target robot itself.

Besides the aforementioned experiments, this work compares actor-critic algorithms TD3, SAC and TQC for continuous control, and studies benefits of several ablations and configurations such as the use of demonstrations, curriculum learning and proprioceptive observations.
