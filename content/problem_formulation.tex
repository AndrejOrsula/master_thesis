\chapter{Problem Formulation}\label{ch:problem_formulation}

This chapter systematically formulates the targetted task of robotic grasping as an MDP, while describing the applied reward function alongside the utilised observation and action spaces. The corresponding implementation of such formulation with detailed specifications is covered in the next chapter \autoref{ch:implementation}.

In this work, the agent is assumed to be a high-level controller that provides sequential decision making in form of gripper pose and its actuation state. Therefore, the environment is considered to not only include all objects and the physical interactions between them but also the robot with its actuators and low-level controllers. Episodic formulation of the grasping task is studied in this work, where a new set of objects is introduced into the scene at the beginning of each episode when the environment is reset. During each episode, the aim of agent is to grasp and lift an object certain height above the ground plane that the objects rest on, which also terminated the current episode. Furthermore, an episode is also terminated after a fixed number of time steps and whenever the agent pushes all objects outside the union of the perceived and reachable workspace. Placing of objects after their picking is not investigated in this work.


\section{Observation Space}

\subsection{Octree}

\subsection{Proprioceptive Observations}

\subsection{Observation Stacking}



\section{Action Space}

% Mention this here
% Some problems are better solved with other traditional method instead of RL. (when talking about control of joint vs cartesian pose)


\section{Reward Function}



\section{Curriculum and Demonstrations}


